\documentclass[a4paper,10pt,sans,colorlinks=true]{moderncv}
\usepackage[english]{babel}
\usepackage[T1]{fontenc}
\usepackage[utf8]{inputenc}
\usepackage{lmodern}
\usepackage{microtype}

\firstname{Mattijs}
\familyname{Korpershoek}
\title{\Large{Embedded Linux Kernel/Android Software Engineer}}
\address{16 Rue Henri Lavigne - Apt 3}{31300, Toulouse}{France}
\phone[mobile]{+33~6~28~30~21~88}
\email{mattijs.korpershoek@gmail.com}
\social[linkedin]{mattijskorpershoek}
\social[github]{makohoek}
%\photo{example.jpg}

\moderncvstyle{fancy}
\moderncvcolor{red}                               % color options 'blue' (default), 'orange', 'green', 'red', 'purple', 'grey' and 'black'
\definecolor{color1}{HTML}{ee0000}% red
\definecolor{gray-entry}{HTML}{383838}% gray
\definecolor{color2}{HTML}{383838}% gray
\nopagenumbers{}
\setlength{\hintscolumnwidth}{50mm}
%\setlength{\hintscolumnwidth}{15mm}
% adjust the page margins
\usepackage[scale=0.85]{geometry}
\AtBeginDocument{\recomputelengths}
\moderncvicons{awesome}


\begin{document}

\hypersetup{urlcolor=gray-entry}

\renewcommand*{\titlestyle}[1]{{\titlefont\textcolor{color2}{#1}}}

\maketitle
\definecolor{links-content}{HTML}{0875E1}% standard blue
\hypersetup{
  urlcolor=links-content,
%  urlbordercolor=links-content,
%  linkbordercolor={0 0 0},
%  pdfborder={0 0 1},
%  pdfborderstyle={/S/U/W 1} % Underline links (thickness: 1pt)}
}

\renewcommand*{\subsectionstyle}[1]{{\large\bfseries\color{gray-entry}{#1}}}

\addvspace{5ex}

\section{Summary}
\begin{cvcolumns}
  \cvcolumn{}{%
    I am an embedded Linux Kernel/Android Software Engineer with 10+ years of experience.
    I have contributed to multiple open source projects such as the Linux Kernel, U-Boot, Libcamera, Android Open Source Project (AOSP) and more.
    I have ported various Android versions (9 to 15) on multiple ARM boards with different SoCs including MediaTek, Texas Instruments and Amlogic.
  }
\end{cvcolumns}

\addvspace{3ex}

\section{Open source contributions}
\begin{cvcolumns}
  \cvcolumn{}{%
    \begin{itemize}
    \item \href{https://lore.kernel.org/lkml/?q=mkorpershoek\%40baylibre.com}{Linux Kernel}
    \item \href{https://android-review.googlesource.com/q/owner:mkorpershoek@baylibre.com+AND+status:merged}{Android (AOSP)}
    \end{itemize}
  }
  \cvcolumn{}{%
    \begin{itemize}
    \item \href{https://patchwork.ozlabs.org/project/uboot/list/?series=&submitter=77589&state=3&q=&archive=&delegate=}{U-Boot}
    \item \href{https://patchwork.libcamera.org/project/libcamera/list/?series=&submitter=153&state=3&q=&archive=&delegate=}{Libcamera}
    \end{itemize}
  }
  \cvcolumn{}{%
    \begin{itemize}
    \item \href{https://github.com/mesonbuild/meson/pull/12361}{Meson}
    \end{itemize}
  }
\end{cvcolumns}

\addvspace{3ex}

\section{Technical skills}
\begin{cvcolumns}
  \cvcolumn{General}{%
    \begin{itemize}
      \item Linux kernel upstream development
      \item U-Boot (maintainer)
      \item Device Tree
      \item Git and backporting patches
      \item Code review and mentoring
      \item System debugging
      \item CI/CD
    \end{itemize}
  }
  \cvcolumn{Android}{%
    \begin{itemize}
      \item BSP development for various SoC vendors
      \item Multiple ARM board support from scratch
      \item Boot flow on ARM boards (U-Boot)
      \item Android Common Kernel (bazel)
      \item HALs (including AIDL)
      \item OTA, multimedia, bringup
    \end{itemize}
  }
\end{cvcolumns}

\addvspace{5ex}

\section{Key accomplishments}
\begin{cvcolumns}
  \cvcolumn{Speaker at ELC in Open Source Summit Europe 2024}{%
    In a 40 minutes \href{https://osseu2024.sched.com/event/1ej1D/how-to-enable-android-aosp-on-your-developer-board-mattijs-korpershoek-baylibre}{talk} named "How to Enable Android (AOSP) on Your Developer Board", I walk through what it takes to run a modern Android (14) on the BeaglePlay ARM development board.
  }
\end{cvcolumns}

\addvspace{1ex}

\begin{cvcolumns}
  \cvcolumn{U-Boot maintainer/custodian (DFU, usb gadget, Android)}{%
    In 2023, after contributing multiple USB fixes, I've been asked to help co-maintaining multiple topics in U-Boot, such as USB gadget, DFU, and some Android related commands. Since then, I've been actively improving the Android support and done \href{https://lore.kernel.org/u-boot/?q=mkorpershoek\%40baylibre.com}{lots of testing and reviewing} for each U-Boot release.
  }
\end{cvcolumns}

\addvspace{1ex}

\begin{cvcolumns}
  \cvcolumn{Founding member of the aosp-devs community}{%
    Android BSP developers did not have a neutral public place to hang out, discuss technical issues or share knowledge. I am a founding member of the \href{https://aosp-devs.org/about}{aosp-devs} initiative: a community for AOSP developers with over 300 members.
  }
\end{cvcolumns}

\addvspace{1ex}

\begin{cvcolumns}
  \cvcolumn{Android multimedia support on Texas Instruments Sitara boards}{%
    Texas Instruments provides Android support for their Sitara AM62X and AM62P SoCs. I have enabled key multimedia features such as \href{https://git.ti.com/cgit/android/external-libcamera/log/?h=d-android15-release\&qt=author\&q=mkorpershoek}{CSI camera support} and \href{https://git.ti.com/cgit/android/external-v4l2\_codec2/log/?h=d-android15-release\&qt=author\&q=mkorpershoek}{video encode/decode acceleration} in Android. I'm also responsible for the Android 15 migration and helped with backporting patches from mainline to the Android Common Kernels (6.6)
  }
\end{cvcolumns}

\section{Work history}
\subsection{Jan, 2020 -- Current}
\cventry{}{Linux Kernel/Android software engineer}{BayLibre}
{Toulouse/remote}{France}
{%
  \begin{itemize}
  \item Worked remotely in small teams of 2-3 engineers
  \item Open-source advocate internally, Android technical leadership
  \item All-around general-purpose engineering work from bootloaders to apps
  \item Android BSP development for multiple SoC vendors
  \item Texas Instruments (Sitara)
    \begin{itemize}
    \item Main developer for TI's \href{https://software-dl.ti.com/processor-sdk-android/esd/AM62PX/10_01_00/docs/devices/AM62PX/android/Release_Specific_Release_Notes.html}{Android SDK}
    \item Development from scratch (AOSP only)
    \item Upstream first strategy, working closely with TI's kernel team
    \item Released Android versions 11 -- 15
    \item Enable new hardware features such as display support, video decoding, CSI camera
    \item Keep up to date with latest Android requirements: HAL migration to AIDL
    \item Track upstream U-Boot and backport/submit features
    \item Android tablet, Android Automotive versions
    \item Monitor automated testing and improve CTS/VTS coverage
    \item Brought up a community board (\href{https://www.beagleboard.org/boards/beagleplay}{BeaglePlay})
    \end{itemize}
  \item Amlogic
    \begin{itemize}
    \item Worked on \href{https://source.android.com/docs/setup/create/devices\#vim3board}{Khadas VIM3} Android reference board
    \item AOSP upstreaming and maintenance
    \item Upstreamed 30+ patches to U-Boot for implementing Android bootflow
    \item Added board support in \href{https://android-review.googlesource.com/c/kernel/common/+/3296472}{Android Common Kernel}
    \item Worked with Android TV team for debugging or new features
    \end{itemize}
  \item MediaTek:
    \begin{itemize}
    \item Started \href{https://gitlab.baylibre.com/baylibre/mediatek/rita/device/mediatek}{MediaTek Genio} support from scratch
    \item Used ChromeOS kernel as baseline
    \item Supported multiple customers to bring out their products using this BSP
    \item Started from board bring-up to customer mass-production
    \item Provided technical training/support to customers
    \item Low maintenance Android BSP, merging most things from upstream
    \item Upgraded to major kernel versions multiple times
    \end{itemize}
  \end{itemize}
}

\subsection{Feb, 2017 -- Dec, 2019}
\cventry{}{Android platform engineer}{Intel (Celad)}
{Santa Clara}{USA}
{%
  \begin{itemize}
  \item Android platform engineer on Android Wear products with Intel inside
  \item Active during entire product life-cycle from the early schematics to the end-user software releases.
  \item Several roles including platform developer, system debugger, and factory line support.
  \item Technical skills:
    \begin{itemize}
    \item Android/Linux device drivers (ASoC), intel platform drivers
    \item Android frameworks: audio HALs, other HALs, Treble
    \item Android apps: mainly debugging, no development.
    \item Production line support, system debugging, hardware bring-ups
    \item Linux, C, C++, Java, Git, Python, XML, Bash, Android build system
    \end{itemize}
  \end{itemize}
}

\subsection{Sept, 2014 -- Jan, 2017}
\cventry{}{Android platform engineer}{Intel (Celad)}
{Toulouse}{France}
{
    \begin{itemize}
    \item Android platform engineer on Android Wear products with Intel inside
    \item Implemented Android Audio stack (HAL, linux machine driver (ASoC)) for digital microphone (voice recognition)
    \item Developed a low-power blue-tooth audio playback architecture (A2DP)
    \item Designed a custom Android ROM used for mass production testing and quality screening in the factory.
    \item Technical skills:
      \begin{itemize}
        \item Android/Linux device drivers (ASoC) and DSP (Tensilica cores) Firmware
        \item Android frameworks: audio HAL, AudioFlinger, AudioPolicy
        \item Production line support, system debugging, hardware bring-ups
        \item Linux, C, C++, Git, Python, XML, Bash, Android build system
      \end{itemize}
    \end{itemize}
}

\subsection{Apr, 2014 -- Aug, 2014}
\cventry{}{Android Audio developer (internship)}{Intel (Celad)}
{Toulouse}{France}
{
  \begin{itemize}
    \item Open-sourced the Parameter-Framework, a major component of Intel's Android Audio HAL
    \item Component is now part of Android AOSP (/external/parameter-framework/)
    \item Extended audio stack on Intel phone reference platforms (dual sim)
    \item Technical skills:
    \begin{itemize}
      \item Linux, C, C++ Git, XML, markdown, Android platform
    \end{itemize}
  \end{itemize}
}

\subsection{May, 2013 -- Aug, 2013}
\cventry{}{Windows developer (internship)}{24green}
{Vlaardingen}{The Netherlands}
{
  \begin{itemize}
    \item Developed a web (REST) API which handles climate control functions in green-houses.
    \item Created apps to illustrate the API.
    \item Technical skills:
    \begin{itemize}
      \item C\#, Windows XP embedded
      \item javascript, html, CSS, java
    \end{itemize}
  \end{itemize}
}

\subsection{Apr, 2012 -- Jul, 2012}
\cventry{}{Linux kernel developer (internship)}{IRIT}
{Toulouse}{France}
{
  \begin{itemize}
    \item Linux kernel development around performance counters and other hardware metrics
    \item Modules developed used by research lab for experiments on CPU frequency (power savings)
    \item Real world testing on a French research grid computing network, Grid'5000
    \item Technical skills:
      \begin{itemize}
        \item Linux kernel, device drivers, C, GNU make, Bash
      \end{itemize}
  \end{itemize}
}

\addvspace{5ex}

\section{Education}
\subsection{2009 -- 2014}
\cventry{}{CAMSI Master's degree}{Paul Sabatier University}
{Toulouse}{France}
{
  \begin{itemize}
  \item CAMSI: Informatics, Systems and Machine Architecture Concepts
  \item rank 1/18 (Valedictorian)
  \item overall score > 85\%
  \end{itemize}
}


\end{document}
